\chapter{Introduction}
\label{sec:introduction}

Quantum chemistry ab initio methods form the foundation for understanding the distribution and arrangement of atoms and their respective electron distributions in molecules. Regardless of the method employed, the calculation must be started with an initial estimate of the electron density. This serves as an initial reference for the self-consistent procedure, that iteratively refines the density to minimize the system's total energy under the constraints of the Schrödinger equation. These self consistent field (SCF\footnote{In the context of this work SCF doesn't exclusively refer to the Hartree-Fock method but as a general term referring to both Hartree-Fock and Kohn-Sham methods.}) methods scale very poorly with respect to the number of basis functions used to expand the multi-electron-wavefunction. Tremendous progress, both in theory, as well as in algorithmic design and compute power, has been made in the last decades. Nevertheless, the scaling of SCF calculations remains of order $\bigO{n^4}$ - $\bigO{n^7}$ (depending on the specific method used) in terms of number of basis functions $n$. Especially for large and poorly converging systems, this still constitutes a bottleneck. \\

Improving the initial guess can drastically cut down the number of SCF iterations needed to reach self-consistency. Most available schemes, whether heuristic shortcuts or minimal-basis atomic projections, produce wildly different convergence behaviors. A high-quality guess not only speeds up convergence but can also be the difference between a successful calculation and one that fails to converge.\\
Because molecular simulations underpin R\&D across pharmaceuticals, materials science, and chemical engineering, faster SCF convergence translates directly into lower computational cost and shorter project timelines. In these industries, where thousands of SCF-calculations may be launched, even millisecond-scale improvements per calculation can add up to enormous time savings, thus reducing to  both energy consumption and financial expenses incurred. \\
\TODO{AIM - why aim - why ml}\\
This Thesis aims to use statistical and machine learning methods to predict an initial guess for the electron density. 
\TODO{Outline of thesis}


