\chapter{Introduction}
\label{sec:introduction}

Quantum chemistry ab initio methods form the foundation for understanding the distribution and arrangement of atoms and their respective electron distributions for molecules. Regardless of the method employed, the calculation must begin with an initial estimate of the electron density. This serves as the starting point for the self-consistent procedure, that iteratively refines the density to minimize the system's total energy under the constraints of the Schrödinger equation. These self consistent field (SCF\footnote{In the context of this work SCF doesn't exclusively refer to the Hartree-Fock method but as a general term referring to both Hartree-Fock and Kohn-Sham methods.}) methods scale very poorly with the number of basis functions used for the expansion of the multi-electron-wavefunction. Although huge progress, both in theory, as well as in algorithmic design and compute power, has been made in the last decades, the scaling of SCF calculations is of $\bigO{n^4}$ - $\bigO{n^7}$ (depending on the specific method used) with the number of basis functions $n$. Especially for large and poorly converging systems, this remains a bottleneck. \\

One way of addressing this problem is to make a more informed and ultimately better initial guess, in order to reduce the number of iterations needed to converge to a self-consistent solution. While there are many initialization schemes available, they are either mostly based on heuristics or simulations on the constituent atoms. While these methods perform adequately for many systems, they sometimes fail to converge or do so poorly for certain molecule. \\
This Thesis aims to use statistical and machine learning methods to predict an initial guess for the electron density, which yields faster convergence. \TODO{Add what was done}. 


