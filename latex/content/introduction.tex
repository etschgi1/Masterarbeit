\chapter{Introduction}
\label{sec:introduction}

Quantum-chemistry \textit{ab initio} methods form the foundation for understanding the distribution and arrangement of atoms and their respective electron distributions in molecules. Regardless of the specific method, every calculation must be started with an initial estimate of the electron density. This serves as an initial reference for the self-consistent procedure, that iteratively refines the density to minimize the system's total energy under the constraints of the Schrödinger equation. These self-consistent field (SCF\footnote{In this work, SCF does not refer exclusively to Hartree-Fock but is used as a general term encompassing both Hartree-Fock and Kohn-Sham methods.}) methods scale very poorly with respect to the number of basis functions used to expand the multi-electron-wavefunction. Tremendous progress, both in theory and in algorithmic design and compute power, has been made in the last decades. Nevertheless, the scaling of SCF calculations remains of order $\bigO{K^4}$ - $\bigO{K^7}$ (depending on the specific method used) in terms of number of basis functions $K$. Especially for large and poorly converging systems, this still constitutes a bottleneck. \\

Improving the initial guess can drastically cut down the number of SCF iterations needed to reach self-consistency. Most available schemes, whether heuristic shortcuts or minimal-basis atomic projections, produce wildly different convergence behaviours. A high-quality guess not only speeds up convergence, but can mean the difference between a successful calculation and one that fails to converge. Because molecular simulations underpin R\&D across pharmaceuticals, material science, and chemical engineering, faster SCF convergence translates directly into lower computational cost and accelerated development cycles. In these industries, where thousands of SCF-calculations may be launched, even millisecond-scale improvements per calculation can add up to enormous time savings, reducing both energy consumption and financial expenses. \\
This thesis employs statistical and machine learning (ML) techniques to predict an initial guess for the electron density. We explore the viability of various ML methodologies to derive the density straight from the overlap matrix or via an intermediate Fock matrix. All schemes will operate directly on the atomic basis functions, rather than projecting them into a chosen computational basis, as in several classical methods. To our knowledge, no such ML-based guessing scheme has been devised and benchmarked against established guessing methods on a large, diverse dataset. Given the rapid advances and broad applicability of ML models, we expect data-driven approaches to uncover subtleties in electronic correlations that traditional schemes, constrained by approximations or minimal basis expansions, cannot capture.\\

\autoref{sec:background} reviews SCF theory and ML methodologies, then introduces the dataset we use. In \autoref{chap:fock_matrix_predictions}, we develop density-guessing schemes based on the Fock matrix in a small basis set and extend them to a larger basis. A graph neural network (GNN) approach is devised in \autoref{chap:gnn} and later applied to various datasets in \autoref{chap:application}. Finally, \autoref{chap:conclusion} offers closing remarks and an outlook.



