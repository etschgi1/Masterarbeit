\chapter{Application}
\label{chap:application}
The GNN model introduced in \autoref{chap:gnn} will be benchmarked on various datasets in the following chapters. 

\TODO{Refer to 0d benchmark model}
\section{QM9 - \ch{C7H10O2} Isomers}
\label{sec:qm9_isomers_benchmark}
\TODO{Notes on training}

Run 03/07
\begin{verbatim}
    Best hyperparameters found were:  {'batch_size': 32, 
    'hidden_dim': 128, 'message_passing_steps': 4, 'edge_threshold_val': 3.048869727327686, 
    'message_net_dropout': 0.2216763962125841, 'data_aug_factor': 1.0, 'message_net_layers': 3, 
    'lr': 0.0026826922264871233, 'weight_decay': 1.7769044544789348e-05, 'num_epochs': 30, 
    'grace_epochs': 5, 'lr_factor': 0.5, 'lr_patience': 3, 'lr_threshold': 0.001, 'lr_cooldown': 2, 
    'lr_min': 1e-06}
    Results saved to /home/ewachmann/REPOS/Masterarbeit/3_studies/Block_guessing/6-31g_testing/tune_results/results_hyp_small.py_20250703_123357.json 
    and best result to /home/ewachmann/REPOS/Masterarbeit/3_studies/Block_guessing/6-31g_testing/tune_results/best_hyp_small.py_20250703_123357.json
    Ray Tune run completed successfully.
\end{verbatim}


\section{QM9 - full?}
\label{sec:qm9_isomers_benchmark}
\TODO{Notes on training}

\section{MD ? }
\label{sec:qm9_isomers_benchmark}
\TODO{Notes on training}
